Welcome to ACL 2019, the 57th Annual Meeting of the Association for Computational Linguistics. This year the conference will be held on July 28–August 2, in the captivating city of Florence. This is the first time that the ACL annual conference visits Italy, and for this occasion, it has selected Florence, the capital of Italy’s Tuscany region, and one of the most beautiful and visited art cities in the world.
ACL 2019 received a record number of papers, close to three thousand, representing a sharp increase with respect to last year. At the moment of writing this letter, we are also expecting record attendance, with close to three thousand participants. This is a clear indicator of how vibrant and dynamic our field is at the moment. At the same time, this continuous growth poses a challenge to the organizers, who have to adapt quickly to numbers that surpass any previous estimation they had. I must say that the team of organizers worked very hard and very professionally in this complex scenario, with the goal of offering a conference program and setting that suit most of the participants. ACL needs to be large in many aspects, but at the same time it should be an enjoyable conference for all, and it should retain the original spirit of the ACL annual conferences. I hope the effort made will bring us close to this difficult double objective.
As usual, the conference will span a six-day period, and it will include a varied program with 9 tutorials (July 28), 18 one-day workshops (August 1-2), the co-located Fourth Conference on Machine Translation (WMT19; August 1-2), the third Widening NLP workshop (WiNLP 2019; July 28), and the exciting main ACL program (July 29-31), which this year will present a record number of 660 original papers. Addition- ally, the main program of ACL will feature the Student Research Workshop and System Demonstrations, with 72 and 34 presentations, respectively. Finally, apart from the papers, ACL 2019 will enjoy also the contribution from two exceptional keynote speakers, Pascale Fung and Liang Huang, and it will see an ACL award ceremony with the Lifetime Achievement Award, the Distinguished Service Award, and the Test-of-Time paper awards.
Each ACL conference is the culmination of a long process, which involves a large team of committed people. It is an honor for me to have coordinated such a team of talented people, who kindly volunteered their time to make this conference possible. I would like to thank each and every one of them. The Program Co-Chairs, Anna Korhonen and David Traum, did a superb job at managing the avalanche of submissions, putting together the program committee, and leading the paper selection process. The Local Co-Chairs, Alessandro Lenci, Bernardo Magnini, and Simonetta Montemagni, were crucial in coordinating with the PCO for all the local arrangements, which were very complex given the growing size of the conference. Fortunately, we all had the help and advice from Priscilla Rasmussen, the ACL Business Manager, who knows everything about ACL conferences, and how to make them a success. The ACL Executive Committee has also been very supportive all the time, providing timely guidance and help to solve the problems that arose in the way.
I want to thank all of the other chairs for their dedication and hard work, more than often under a tight schedule: Workshop Co-Chairs Barbara Plank and Sebastian Riedel; Tutorial Co-Chairs Preslav Nakov and Alexis Palmer; Demo Co-Chairs Enrique Alfonseca and Marta R. Costa-jussà; Student Re- search Workshop Co-Chairs Fernando Alva-Manchego, Eunsol Choi and Daniel Khashabi; SRW Fac- ulty Advisors Hannaneh Hajishirzi, Aurelie Herbelot, Scott Wen-tau Yih and Yue Zhang; Publication Co-Chairs Douwe Kiela, Ivan Vulic, Shay Cohen and Kevin Gimpel; Conference Handbook Co-Chairs Elena Cabrio and Rachele Sprugnoli; Conference App Chair Andrea Cimino; Local Arrangement Commit- tee Sara Goggi, Maria Cristina Schiavone, Sacha Bourdeaud’Hui; Local Sponsorship Co-Chairs Roberto Basili and Giovanni Semeraro; Publicity Co-Chairs Felice Dell’Orletta, Lucia Passaro and Sara Tonelli; Mentorship Co-Chairs Rada Mihalcea, Robert Frederking and Aakanksha Naik; and Student Volunteer Coordinators Dominique Brunato, Marco Senaldi and Giulia Venturi.
I am also very grateful to the chairs of the previous years’ conferences (not only ACL but also NAACL and EMNLP), who were always ready to help and to provide advice, contributing to the transmission, from year to year, of all the know-how and collective memory.
Many thanks to the senior area chairs, the area chairs, the reviewers, our workshop organizers, our tutorial instructors, the authors and presenters of papers, and the invited speakers.
I am also deeply grateful to all the sponsors for their great support to the conference.
Finally, I would like to thank all the participants, who will be the main actors from July 28 to August 2, 2019. I am convinced that we will experience a fantastic conference, scientifically exciting and full of fond memories, in the unique environment of Florence. Looking forward to seeing all of you there!

ACL 2019 General Chair \\
Lluís Màrquez, Amazon

